\documentclass[a4paper]{article}
\usepackage[ps2pdf]{thumbpdf}
\usepackage{graphicx}   
\usepackage{listings}
\usepackage{palatino}
\usepackage{color}
\usepackage{textcomp}

\usepackage[
  ps2pdf,
  pagebackref,
  pdfpagelabels,
  extension=pdf,
]{hyperref}


\hypersetup{ 
  pdftitle          = {CS6360 Project},
  pdfsubject        = {CS6360 Project},
  pdfauthor         = {Immanuel Ilavarasan Thomas,Kirtika Ruchandani},
  pdfkeywords       = {},
  pdfcreator        = {ps2pdf with Ghostscript},
  pdfproducer       = {LaTeX with hyperref and thumbpdf},
  bookmarks         = true,
  bookmarksopen     = true,
  bookmarksnumbered = true,
  pdfstartpage      = {1},
  pdfpagemode       = UseThumbs,
  colorlinks        = true,
  linkcolor         = red,
  anchorcolor       = red,
  citecolor         = blue,
  filecolor         = red,
  pagecolor         = red,
  urlcolor          = red
}


% Colors
\definecolor{gold}{rgb}{0.85,.66,0}
\definecolor{db}{rgb}{0,0.08,0.45} 
\definecolor{lb}{rgb}{0.8,0.85,1}
\definecolor{llb}{rgb}{0.90,0.90,1}
\definecolor{grey}{gray}{0.4}

\lstset{
        language=C++,
        keywordstyle=\bfseries\ttfamily\color[rgb]{0,0,1},
        identifierstyle=\ttfamily,
        commentstyle=\color[rgb]{0.133,0.545,0.133},
        stringstyle=\ttfamily\color[rgb]{0.627,0.126,0.941},
        showstringspaces=false,
        basicstyle=\small,
        numberstyle=\footnotesize,
        numbers=left,
        stepnumber=1,
        numbersep=10pt,
        tabsize=2,
        showspaces=false,
        showstringspaces=false,
        showtabs=false,
        breaklines=true,
        prebreak = \raisebox{0ex}[0ex][0ex]{\ensuremath{\hookleftarrow}},
        breakatwhitespace=false,
        breaklines=true,
        aboveskip={1.5\baselineskip},
        columns=fixed,
        upquote=true,
        extendedchars=true,
        frame=single,
        backgroundcolor=\color{llb},
}

\newcommand{\emp}[1]{\textbf{\emph{#1}}}



% Page layout
\oddsidemargin 0.0in
\textwidth 6.5in
\headheight 0.0in
\topmargin 0.0in
\textheight 9.2in
\headsep 0.0in

% Macros
\def \defbox #1{
\colorbox{lb}{\parbox{5in}{#1}}
}

\title{\color{db} \textbf{CS6360 Project: Demonstrating of Gaming Features}}
\author{Immanuel Ilavarasan Thomas, CS07B006 \\  Kirtika Ruchandani, CS07B040}


\begin{document}

\sffamily
\maketitle

\section{\color{db}Introduction}



\section{\color{db}Project Statement}
Select any set of tools (gaming engines, libraries etc) and display gaming
features involving one or more human characters. Minimum requirement is to have 
character animation such as walking, running etc user-controlled and to
demonstrate constraints on the motion. 

\subsection{\color{db}Overview of the submitted project}
We present a maze-traversal game with the following features -
\begin{itemize}
\item Scenario that has a Skybox, tiled-terrain and textured walls. All of these are made using OpenSceneGraph.
\item Character animation - the protagonist can stand still, walk around and run. The animations use Cal3D.
\item Physics Engine - We detect collision between the model and the walls, the power boosters and take appropriate action.
\item Dynamic Maze generation - we generate a new maze each time using a python script and add/remove power boosters dynamically.
\item Callback-based, event-driven execution - we use callbacks to process events such as collision detection. Callbacks are also used in the Bullet Physics framework to process events on every simulation tick.
\item Heads Up Display - to provide on-screen help to the player and to display the score.
\item Initial Splash Screen - this is done using OpenSceneGraph's support for switches.
\end{itemize}




\section{\color{db}Background}
\subsection{\color{db}Libraries used}
\begin{itemize}
\item \textbf{OpenSceneGraph - v.2.8.0 : } Library for rendering scene data - terrain and skyboxes 
\item \textbf{Cal3d - v0.11 : } Library for character models (mesh data) and animation control
\item \textbf{BulletPhysics - v2.77 : } Physics engine for collision detection and dynamics implementation
\end{itemize}




\section{\color{db}Work Done}
\subsection{\color{db}Details of the game graphics}
\subsubsection{\color{db}The SceneGraph flowchart}

\subsection{\color{db}Details of the physics backend}
Bullet Physics is a dynamics engine that provides support for collision detection, rigid body dynamics and constrained motion. In the beginning of the program, we create a \emp{btDynamicsWorld} - this is the encompassing entity that keeps tracks of all the physics objects in the system, performs the simulation each cycle and updates the position and velocity values of all the rigid bodies. \\
A \emp{btRigidBody} is created corresponding to each object we are interested in. In this case, we have a btRigidBody corresponding to the model, each of the walls and the power-ups.  \\
The \emp{btDynamicsWorld} includes a \emp{btAxisSweep} broadphase, a constraint solver and collision dispatcher. The dispatcher keeps a list of all colliding objects at any simulation instant which can be queried.


\subsubsection{\color{db}Updating the scene every instant}
On very simulation tick, Bullet Physics looks up the previous configuration of the rigid body corresponding to the OSG body in the graph. It needs to know the initial positions and velocities of all bodies. The dispatcher uses custom collision-detection algorithms for each pair of bodies in the list and then the collisions can be processed. The updated position and velocities are made available by Bullet in the form of a GL matrix which can be passed on to OSG to draw the corresponding object.
\begin{lstlisting}

btDefaultMotionState* myMotionState = (btDefaultMotionState*) _body->getMotionState();
myMotionState->m_graphicsWorldTrans.getOpenGLMatrix(m);

osg::Matrixf mat(m);

osg::PositionAttitudeTransform *pat = dynamic_cast<osg::PositionAttitudeTransform *> (node);
pat->setPosition(mat.getTrans());
pat->setAttitude(mat.getRotate());

osg::Vec3 position = pat->getPosition();

\end{lstlisting}

\subsubsection{\color{db}Collision Detection}



% Phart on how bullet physics interacts with OSG  with example
% How it updates the coordinates and motion info for OSG to draw
% Collision detection and callbacks that get activated


\section{\color{db}Results}
\subsection{\color{db}Screenshots}
Lorem Ipsum
% One for the starting screen
% One for the character animations
% One for within the maze - preferably with a powerup on top


\section{\color{db}Conclusion}
\section{\color{db}References}




\end{document}

