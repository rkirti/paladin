\documentclass[a4paper]{article}
\usepackage[ps2pdf]{thumbpdf}
\usepackage{graphicx}   
\usepackage{listings}
\usepackage{palatino}
\usepackage{color}

\usepackage[
  ps2pdf,
  pagebackref,
  pdfpagelabels,
  extension=pdf,
]{hyperref}


\hypersetup{ 
  pdftitle          = {CS6360 Project},
  pdfsubject        = {CS6360 Project},
  pdfauthor         = {Immanuel Ilavarasan Thomas,Kirtika Ruchandani},
  pdfkeywords       = {},
  pdfcreator        = {ps2pdf with Ghostscript},
  pdfproducer       = {LaTeX with hyperref and thumbpdf},
  bookmarks         = true,
  bookmarksopen     = true,
  bookmarksnumbered = true,
  pdfstartpage      = {1},
  pdfpagemode       = UseThumbs,
  colorlinks        = true,
  linkcolor         = red,
  anchorcolor       = red,
  citecolor         = blue,
  filecolor         = red,
  pagecolor         = red,
  urlcolor          = red
}


%\def\codefont{
%  \fontspec{Courier New}
%  \fontsize{9pt}{11pt}\selectfont}
%\newenvironment{code}
%{\begin{center}
%    \begin{tikzpicture}
%      \node [fill=lb,rounded corners=5pt]
%      \bgroup
%      \bgroup\codefont
%      \begin{tabular}{l}}
%      {\end{tabular}
%      \egroup
%      \egroup;
%    \end{tikzpicture}
%  \end{center}}

% Colors
\definecolor{gold}{rgb}{0.85,.66,0}
\definecolor{db}{rgb}{0,0.08,0.45} 
\definecolor{lb}{rgb}{0.8,0.85,1}
\definecolor{grey}{gray}{0.4}




\lstset{ %
language=C,                % choose the language of the code
basicstyle=\footnotesize,       % the size of the fonts that are used for the code
numbers=left,                   % where to put the line-numbers
numberstyle=\footnotesize,      % the size of the fonts that are used for the line-numbers
stepnumber=2,                   % the step between two line-numbers. If it's 1 each line 
                                % will be numbered
numbersep=5pt,                  % how far the line-numbers are from the code
backgroundcolor=\color{lb},  % choose the background color. You must add \usepackage{color}
showspaces=false,               % show spaces adding particular underscores
showstringspaces=false,         % underline spaces within strings
showtabs=false,                 % show tabs within strings adding particular underscores
frame=single,	                % adds a frame around the code
tabsize=2,	                % sets default tabsize to 2 spaces
captionpos=b,                   % sets the caption-position to bottom
breaklines=true,                % sets automatic line breaking
breakatwhitespace=false,        % sets if automatic breaks should only happen at whitespace
escapeinside={\%*}{*)},         % if you want to add a comment within your code
morekeywords={*,...}            % if you want to add more keywords to the set
}





% Page layout
\oddsidemargin 0.0in
\textwidth 6.5in
\headheight 0.0in
\topmargin 0.0in
\textheight 9.2in
\headsep 0.0in

% Macros
\def \defbox #1{
\colorbox{lb}{\parbox{5in}{#1}}
}

\title{\color{db} \textbf{CS6360 Project: Demonstrating of Gaming Features}}
\author{Immanuel Ilavarasan Thomas, CS07B006 \\  Kirtika Ruchandani, CS07B040}


\begin{document}

\sffamily
\maketitle



\section*{\color{db}Project Statement}
Select any set of tools (gaming engines, libraries etc) and display gaming
features involving one or more human characters. Minimum requirement is to have 
character animation such as walking, running etc user-controlled and to
demonstrate constraints on the motion. 

\subsection*{\color{db}Overview of the submitted project}
Lorem Ipsum
% Features covered -
% Skybox, terrain and textured walls
% Character animation -- the model can stand still, walk around or run
% Physics Engine - collision detection - model cannot walk into walls, power boosters blown up on collision
% Overhead Display - we display score

\subsection*{\color{db}Libraries used}
\begin{itemize}
\item \textbf{OpenSceneGraph - v.2.8.0 : } Library for rendering scene data - terrain and skyboxes 
\item \textbf{Cal3d - v0.11 : } Library for character models (mesh data) and animation control
\item \textbf{BulletPhysics - v2.77 : } Physics engine for collision detection and dynamics implementation
\end{itemize}

\section*{Details of the game graphics}
\subsection*{The SceneGraph flowchart}
\subsection*{Screenshots}
Lorem Ipsum
% One for the starting screen
% One for the character animations
% One for within the maze - preferably with a powerup on top

\section*{Details of the physics backend}
Lorem Ipsum 
% Phart on how bullet physics interacts with OSG  with example

\section*{Code Snippets}
Lorem Ipsum 


\section{\color{db}Result data and code}
\subsection{\color{db}Result Data}
All the results are available in the submitted folder in the files name block$<size>$\_results.txt. The following table presents a comparison for some sizes - \\
\begin{tabular}{| l|l|l|l|}
Size & Blocking & Naive & Speedup \\ 
256  & 0.069356 & 0.069356 & 1.513   \\         
512  & 0.580515 & 0.580515 & 2.645  \\
768  & 1.949473 & 1.949473 & 3.100  \\
1024 &  5.685684 & 5.685684 &  3.70  \\
1280 &  9.447211 & 9.447211 &  4.90  \\ 
1536 &  16.804868 & 16.804868 &  5.085 \\
1792 &  25.983063 & 25.983063 & 5.435 \\
2048 &  56.567231 & 56.567231 & 3.845  \\
2304 &  55.255916 & 55.255916 & 5.707 \\
2560 &  77.798290 & 77.798290 & 5.733 \\

\end{tabular}
\clearpage
\subsection{\color{db}Code}
\begin{lstlisting}

/*
 * Copyright (C) 2010 Kirtika B Ruchandani <kirtibr@gmail.com> 
 * 
 * You may redistribute it and/or modify it under the terms of the
 * GNU General Public License, as published by the Free Software
 * Foundation; either version 2 of the License, or (at your option)
 * any later version.
 * 
 * This program is distributed in the hope that it will be useful,
 * but WITHOUT ANY WARRANTY; without even the implied warranty of
 * MERCHANTABILITY or FITNESS FOR A PARTICULAR PURPOSE.
 * See the GNU General Public License for more details.
 * 
 * You should have received a copy of the GNU General Public License
 * along with this program.  If not, write to:
 * 	The Free Software Foundation, Inc.,
 * 	51 Franklin Street, Fifth Floor
 * 	Boston, MA  02110-1301, USA.
 */

#include <stdio.h>
#include <iostream>
#include <stdlib.h>
#include <sys/time.h>

using namespace std;

#define BLOCK_SIZE 256

class Matrix
{

    public:
        int** matPtr;
        int dimSize;
        int numBlocks;

        // Allocates a matrix and initialises elements to 0
        Matrix(int size)
        { 
            this->dimSize = size;
            this->numBlocks = (this->dimSize)/BLOCK_SIZE;
            this->matPtr  = (int**)malloc(sizeof(int*)*dimSize);
            int j,k;
            for (j=0;j<dimSize;j++)
                *(matPtr + j) = (int*) malloc(sizeof(int)*dimSize);

            for (j=0;j<this->dimSize;j++)
            {
                for (k=0;k<this->dimSize;k++)
                {
                    this->matPtr[j][k] = 0;
                }
            }
        }

        void AddBlockToSelf(Matrix block,int xIdx,int yIdx)
        {
            if (block.dimSize != BLOCK_SIZE)
            {
                cout << "Invalid block" << endl;
                exit(0);
            }

            int j,k;
            for (j=0;j<BLOCK_SIZE;j++)
            {
                for (k=0;k<BLOCK_SIZE;k++)
                {
                    (this->matPtr)[xIdx*BLOCK_SIZE + j][yIdx*BLOCK_SIZE +k] += (block.matPtr)[j][k];
                }
            }
        }



        Matrix GetBlock(int xIdx,int yIdx)
        {
            Matrix result(BLOCK_SIZE);
            result.dimSize = BLOCK_SIZE;
            result.numBlocks = 1;
            int j,k;
            for (j=0;j<BLOCK_SIZE;j++)
            {
                for (k=0;k<BLOCK_SIZE;k++)
                {
                    result.matPtr[j][k]  += (this->matPtr)[xIdx*BLOCK_SIZE + j][yIdx*BLOCK_SIZE +k] ;
                }
            }
            return result;
        }

        void TestPopulate(void)
        {
            int j,k;
            for (j=0;j<this->dimSize;j++)
            {
                for (k=0;k<this->dimSize;k++)
                {
                    this->matPtr[j][k] = (j+k);
                }
            }
        }


        void BlockMultiply(Matrix A, int AXIdx, int AYIdx, Matrix B, int BXIdx, int BYIdx, int resXIdx, int resYIdx)
        {
            if ((A.dimSize != B.dimSize))
            {
                cout << "Sizes dont match." << endl;
                exit(0);
            }

            int j,k,l;


            for (j=0;j<BLOCK_SIZE;j++)
            {
                for (k=0;k<BLOCK_SIZE;k++)
                {
                    for (l=0;l<BLOCK_SIZE;l++)
                    {
                        this->matPtr[resXIdx*BLOCK_SIZE + j][ resYIdx*BLOCK_SIZE + k] += A.matPtr[AXIdx*BLOCK_SIZE + j][ AYIdx*BLOCK_SIZE +l] * B.matPtr[ BXIdx*BLOCK_SIZE + l][ BYIdx*BLOCK_SIZE +k]; 
                    }
                }
            }

        }


        void Print(void)
        {
            int j,k;
            for (j=0;j<this->dimSize;j++)
            {
                for (k=0;k<this->dimSize;k++)
                {
                    cout << this->matPtr[j][k] ;
                }
                fflush(stdout);
                cout << "\n" ;
            }

        }

};



Matrix Multiply(Matrix A, Matrix B)
{
    if (A.dimSize != B.dimSize)
    {
        cout << "Sizes dont match" << endl;
        exit(0);
    }

    Matrix result(A.dimSize);
    int j,k,l;
    for (j=0;j<A.numBlocks;j++)
    {
        for (k=0;k<A.numBlocks;k++)
        {
            for (l=0;l<A.numBlocks;l++)
            {
                result.BlockMultiply(A,j,l,B,l,k ,j,k);
            }    
        }
    } 

    return result;
}


Matrix NaiveMultiply(Matrix A, Matrix B)
{
    if (A.dimSize != B.dimSize)
    {
        cout << "Sizes dont match" << endl;
        exit(0);
    }

    Matrix result(A.dimSize);
    int j,k,l;
    for (j=0;j<A.dimSize;j++)
    {
        for (k=0;k<A.dimSize;k++)
        {
            for (l=0;l<A.dimSize;l++)
            {
                (result.matPtr)[j][k] += (A.matPtr)[j][l]*(B.matPtr)[l][k];
            }    
        }
    } 

    return result;
}





int main(int argc, char **argv)
{
    if (argc != 2)
    {
        cout << "Usage: " << argv[0] << "  <size of matrix>" << endl;
        exit(0);
    }
    Matrix myMat(atoi(argv[1]));
    myMat.TestPopulate();
    
    timeval tim;
    gettimeofday(&tim, NULL);
    double t1=tim.tv_sec+(tim.tv_usec/1000000.0);
    Matrix result = NaiveMultiply(myMat,myMat);
    gettimeofday(&tim, NULL);
    double t2=tim.tv_sec+(tim.tv_usec/1000000.0);
    printf("%6d\t%.6lf\n", atoi(argv[1]),t2-t1);
    //result.Print();
    return 0;
}
\end{lstlisting}






\end{document}

